The documents, which form the basis of this work, are old authorities correspondences where it comes to the storage of nuclear waste. The topics are mainly plans of emergency measures, eliminating access waters and the closure of individual chambers of the mine. At the beginning there were 71 authorities correspondences. 33 of these documents are freely available online. The remaining 38 documents have been provided as part of a research project. The freely accessible documents were written in the years 2009 to 2013. The others, as far as recognizable, are from the years 1971 to 1975. Although picture elements are included in the OCR output, they aren't used in this work. Twelve of the documents were removed because they effectively did not contain any information content. Thus were still 59 documents left with a total of 1239 text areas. In addition, seven empty text areas were removed. Thus remained in 1232 text areas for the following steps. The resulting in the OCR process documents were available in HTML format and has been brought in a CSV format for the next steps which contains the following information for each text area:


%Die Dokumente, welche die Grundlage dieser Arbeit bilden, sind alte Beh\"ordenkorrespondenzen, in denen sich mit der Lagerung von Atomm\"ull befasst wird. Hierbei geht es um Themen wie zum Beispiel die Planung von Notfallmaßnahmen, Beseitigung von Zutrittswässern und die Schließung einzelner Kammern des Bergwerks. Zu Beginn gab es 71 Beh\"ordenkorrespondenzen. Davon sind 33 Dokumente online frei zugänglich. Die restlichen 38 Dokumente wurden im Rahmen eines Forschungsprojektes zur Verf\"ugung gestellt. Die 33 frei zug\"anglichen Dokumente sind aus den Jahren 2009 bis 2013. Die restlichen 38 Dokumente sind, soweit erkennbar, aus den Jahren 1971 bis 1975. Zwar sind auch Bildelemente in der OCR Ausgabe enthalten, jedoch wird auf diese in dieser Arbeit nicht zur\"uckgegriffen. Es wurden zwölf Dokumente entfernt, da sie effektiv keinen Informationsgehalt enthielten. Somit standen noch 59 Dokumente mit insgesamt 1239 Textbereichen zur Verfügung. Zusätzlich wurden sieben leere Textbereiche entfernt. Somit verblieben für die folgenden Schritte 1232 Textbereiche. Die im OCR Prozess entstandenen Dokumenten lagen im HTML-Format vor und wurden für die weiteren Schritte in einen CSV-Format gebracht der für jeden Textbereich folgende Informationen enthielt: 

\begin{itemize}
\setlength{\itemsep}{-5pt}
 \item Document name and index the text area within the document
 \item identifier of the text area assigned classification
 \item position of the text area in the document and its width
 \item by the OCR process recognized text
\end{itemize}

\begin{comment}
\begin{itemize}
\setlength{\itemsep}{-5pt}
 \item Dokumentenname und Index des Textbereiches innerhalb des Dokumentes
 \item Kennung der dem Textbereich zugewiesenen Klassifikation
 \item Position des Textbereiches im Dokument und seine Breite
 \item der vom OCR-Prozess erkannte Inhalt
\end{itemize}
\end{comment}